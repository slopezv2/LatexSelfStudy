\documentclass{article}
\usepackage[dvipsnames]{xcolor}
\begin{document}
	\pagecolor{Goldenrod!50}
	\textcolor{red}{My text is in red color}
	\textcolor{blue}{My text is in blue color}
	\textcolor{red!50!blue}{This is a mixture of red and blue color}
	\textcolor{Plum}{Plum is my favourite fruit}
	\textcolor{WildStrawberry}{Mass is both a property of a physical body and a measure of its resistance to acceleration (a change in its state of motion) when a net force is applied. An object's mass also determines the strength of its gravitational attraction to other bodies.}
	\textcolor{ForestGreen}{The basic SI unit of mass is the kilogram (kg). In physics, mass is not the same as weight, even though mass is often determined by measuring the object's weight using a spring scale, rather than balance scale comparing it directly with known masses. An object on the Moon would weigh less than it does on Earth because of the lower gravity, but it would still have the same mass. This is because weight is a force, while mass is the property that (along with gravity) determines the strength of this force.}
\end{document}