\documentclass[12pt]{article}
\usepackage[margin=1in]{geometry}
\title{Tiger}
\author{Wiki}
\date{\today}

\usepackage{lipsum}
\usepackage{imakeidx}
\makeindex[columns=1, title=Back Index, intoc]

\begin{document}
	\maketitle
	\tableofcontents
	
	\pagebreak
	
	\section{Introduction}
	\paragraph{}
	
	The tiger \index{tiger} (Panthera tigris) \index{Panthera} is the largest living cat species\index{habitat!India} and a member of the genus Panthera\index{habitat!India}. It is most recognisable for its dark vertical stripes on orange-brown fur with a lighter underside. It is an apex predator, primarily preying on ungulates such as deer and wild boar. It is territorial and generally a solitary but social predator, requiring large contiguous areas of habitat, which support its requirements for prey and rearing of its offspring. Tiger cubs stay with their mother for about two years, before they become independent and leave their mother's home range to establish their own.
	
	\section{Lorem Ipsum}
	\lipsum[1-4] 
	
	
	\lipsum[1-4]
	\index{tiger}
	
	\section{Social}
	Young female tigers establish\index{Asia|see{habitat}} their first territories close to their mother's. The overlap between the female and her mother's territory reduces with time. Males, however, migrate further than their female counterparts and set out at a younger age to mark out their own area. A young male acquires territory either by seeking out an area devoid of other male tigers, or by living as a transient in another male's territory until he is older and strong enough to challenge the resident male. Young males seeking to establish themselves thereby comprise the highest mortality rate (30–35\% per year) amongst adult tigers
	
	\printindex
\end{document}